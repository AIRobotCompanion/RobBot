\chapter{Introduction}

\section{Purpose}
The perpose of this document is to describe the requirement specifucation for a robot controlling software for software development team.
\\The intended audience of this document includes the prospective developers of the tool and the technical assessment personnel of the client organization.

\section{Scope}
The name of the software would be RobBot.
\par The RobBot will allow users to controll some very basic operations of a robot comes with the same package, which should have the capability to move, talk, understand very basic commands and environment situations.
\par The software could be used in any OS for PC as well as Android's latest versions. It would have capabilities for enhancing security and authentication systems for manager type users.

\section{Definition, acronyms \& abbrevations}
\begin{itemize}
	\item \textbf{RobBot:} Robotic Chat-Bot.
	\item \textbf{Activity:} nature of what is beign done, like analyzing, coding, testing, \textit{etc.}
	\item \textbf{Artifact:} physical entity that results from an activity, like SRS, architecture diagram, UML diagrams, source code, test scripts, user manual
	\item \textbf{UAM:} user authentication module
	\item \textbf{DAM:} device authentication module
	\item \textbf{UM:} user module
	\item \textbf{MM:} manager module
	\item \textbf{AM:} action module
	\item \textbf{SRS:} software reqiurement specification
\end{itemize}

\section{References}
\subsection{Standard}
IEEE Std. 830-1993: IEEE Recommended Practice for Software Requirements
\subsection{Books}
Java The Complete Reference by Herbert Schildt\\
Understanding the Linux Kernel by D. P. Bovet \& M. Cesati\\
Arduino Cookbook by Michael Margolis
\subsection{Online links}
\url{https://create.arduino.cc/projecthub/jeffpar0721/add-wifi-to-arduino-uno-663b9e}
\\\url{https://forum.arduino.cc/index.php?topic=283043.0}
\\\url{http://www.teomaragakis.com/hardware/electronics/how-to-connect-an-esp8266-to-an-arduino-uno/}
\\\url{}
\section{Overview}
The rest of this document contains an overall description of the RobBot software system (Chapter-2), and the specific requirements for the system(Chapter-3).
